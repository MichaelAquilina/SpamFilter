\section{Alternatives to Naive Bayes}
As a classifier is modeld after some given assumptions and with specific architechtural properities, e.g. all attributes are equally important, it is important to check if the used classifier is the right model for the examined problem.
A very good feature of the Naive Bayes classifier is its ability to train on a dataset with a very high dimensionality in decent time.
Because this property is not shared among all other classifier and some even severely suffer from high dimensionality (the so called Curse of Dimensionality \cite{bellman1957dynamic}), we are using the reduced set of features we estimated in the feature selection process without further research on the impact of other thresholds on feature trimming w.r.t. to non-bayesian classifiers.

\subsection{Decision Trees}

We are going to use Weka's \cite{hall2009weka} implementation of the C4.5 decision tree \cite{Quinlan1993} (known in Weka as J48) as the first alternative to Naive Bayes classification.
C4.5 builts a tree based on the features using information entropy.
The dimension with the highest information gain will be used at each node new to generate a new split.
After the creation of the tree, C4.5 tries to prune all unnecessary branches with leafs to keep the total number of nodes to a minimum.
Altough this feature could be disabled, we are using it to avoid overfitting.

As the runtime of C4.5 increases in a higher level with the number of dimensions than the Naive Bayes Classifier, we are going to use it with the threshold $\alpha=0.017$ and $\beta=0.19$.
This reduced data set of only 1568 dimensions can be used to train a C4.5 tree in a feasible amount of timeThis reduced data set of only 1568 dimensions can be used to train a C4.5 tree in a feasible amount of time.

One of the parameters that can be adjusted is the number of instances each leaf of the tree needs to represent at least during the training phase.
The default setting of WEKA requires at minimum two instances at each leaf, in Figure ?? we outline the scores with varying number of minimum instances.
\todo{Vary minimum number of instances}

\todo{Improvement? Use statistics to show significance (or that we cannot reject the null hypothesis)}


